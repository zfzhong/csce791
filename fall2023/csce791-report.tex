\documentclass[11pt, oneside]{article}   	% use "amsart" instead of "article" for AMSLaTeX format
%\usepackage{geometry}                		% See geometry.pdf to learn the layout options. There are lots.
%\geometry{letterpaper}                   		% ... or a4paper or a5paper or ... 
%\geometry{landscape}                		% Activate for rotated page geometry
%\usepackage[parfill]{parskip}    		% Activate to begin paragraphs with an empty line rather than an indent

\usepackage{geometry}
 \geometry{
 a4paper,
 total={170mm,257mm},
 left=20mm,
 top=22mm,
 bottom=22mm
 }

\usepackage{graphicx}				% Use pdf, png, jpg, or eps§ with pdflatex; use eps in DVI mode
								% TeX will automatically convert eps --> pdf in pdflatex		
\usepackage{amssymb}
\usepackage{amsmath}
\usepackage{fancyhdr}
\usepackage[utf8]{inputenc}
\usepackage[english]{babel}
\usepackage{enumerate}
\usepackage{arcs}
\usepackage{cancel}
\usepackage{xfrac}
\usepackage{amsthm}
\usepackage{gensymb}
\usepackage{xspace}
\usepackage{array}
\usepackage{tabularx}
\usepackage{url}
\usepackage{hyperref}
%\usepackage{ctex}

%SetFonts

%SetFonts

\usepackage[inline]{asymptote}


\pagestyle{fancy}
\fancyhf{}
\rhead{Zifei (David) Zhong, \url{zhongz@email.sc.edu}}
\lhead{\leftmark}
\lfoot{\href{https://github.com/zfzhong/csce791}{github.com/zfzhong/csce791/fall2023}}
\rfoot{\thepage}

\title{CSCE 791 Course Report}
\author{Zifei (David) Zhong}
%\date{}							% Activate to display a given date or no date

\newcommand{\latex}{\LaTeX\xspace}


\begin{document}
\maketitle

This file includes all my course reports for CSCE 791 for the semester
of Fall 2023. I will write a summary for each talk presented in the
seminar. I will also include interesting questions raised by the
audience, as well as corresponding responses from the speaker.\\

\begin{center}
\begin{tabularx}{0.65\textwidth}{r X}
\textbf{Course:} & CSCE 791: Seminar in Advances in Computing\\
\textbf{Location:} & Storey Innovation Center 1400\\
\textbf{Time:} & Friday 2:20pm - 3:10pm\\
\textbf{Semester:} & Fall 2023
\end{tabularx}
\end{center}

\newpage
\section{Talk on August 25, 2023}
Open remarks.

\newpage
\section{Talk on September 1, 2023}
\begin{tabularx} {\textwidth}{r X}
\textbf{Topic}: & AI research in the era of ChatGPT \\
\textbf{Speaker:} & Dr. Biplav Srivastava \\
\end{tabularx}

\subsection{Summary}
Dr. Srivastava introduced the evolving of his AI research projects in the era of ChatGPT. He firstly introduced Plansformer, a system that solves planning tasks. Later, they developed Plansformer in to a new system with architecture call The SOFAI Planner, which incorporates LLM to help solve planning tasks.

Through his research experience, Dr. Srivastava conveyed the message that LLMs are exciting but not reliable, and thus it won't be directly applied to solve practical problems. However, Dr. Srivastava suggested that using LLMs as an efficient way to test research ideas, and then more trustworthy and reliable applications can be built based on what LLMs generate.

He mentioned that ChatGPT responds with different answers for similar questions, and that might cause people not using it because it's behavior is not very predictable (which might be considered as faulty). I would say that for some questions there is no exact answer, and ChatGPT is smart in answering these questions.

\newpage
\section{Talk on September 8, 2023}
\begin{tabularx} {\textwidth}{r X}
\textbf{Topic}: & Resource-Aware Approximate Dynamic Programming and Reinforcement Learning for Optimal Control of Dynamic Cyber-Physical Systems \\
\textbf{Speaker:} & Dr. Avimanyu Sahoo\\
\end{tabularx}

\subsection{Summary}
Dr. Sahoo started his talk by introducing control of dynamic systems, with NASA's Kepler spacecraft as an example. He then proceeded to present the challenges existing in control of cyber-physical systems: bandwidth, computation capability, and uncertainties. He spent most of the talk discussing the possible solutions: adaptive control, neural network control, approximate dynamic programming, and reinforcement learning based optimal control.

Throughout detailed discussion of these solutions, Dr. Sahoo pointed out that these solutions have different resource constraints. Since traditional iterative solutions are computationally expensive, resource-aware control is designed to address the problem. Dr. Sahoo discussed resource-aware self-learning optimal control schemes. 

A lot of mathematical formulas were introduced in the talk. I didn't grasp the details of them. The high level ideas of resource-aware self-learning optimal control are quite interesting. 


\newpage
\section{Talk on September 15, 2023}
\begin{tabularx} {\textwidth}{r X}
\textbf{Topic}: & xxx\\
\textbf{Speaker:} & Dr. yyy\\
\end{tabularx}



\end{document} 
