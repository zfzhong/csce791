\documentclass[11pt, oneside]{article}   	% use "amsart" instead of "article" for AMSLaTeX format
%\usepackage{geometry}                		% See geometry.pdf to learn the layout options. There are lots.
%\geometry{letterpaper}                   		% ... or a4paper or a5paper or ... 
%\geometry{landscape}                		% Activate for rotated page geometry
%\usepackage[parfill]{parskip}    		% Activate to begin paragraphs with an empty line rather than an indent

\usepackage{geometry}
 \geometry{
 a4paper,
 total={170mm,257mm},
 left=20mm,
 top=22mm,
 bottom=22mm
 }

\usepackage{graphicx}				% Use pdf, png, jpg, or eps§ with pdflatex; use eps in DVI mode
								% TeX will automatically convert eps --> pdf in pdflatex		
\usepackage{amssymb}
\usepackage{amsmath}
\usepackage{fancyhdr}
\usepackage[utf8]{inputenc}
\usepackage[english]{babel}
\usepackage{enumerate}
\usepackage{arcs}
\usepackage{cancel}
\usepackage{xfrac}
\usepackage{amsthm}
\usepackage{gensymb}
\usepackage{xspace}
\usepackage{array}
\usepackage{tabularx}
\usepackage{url}
\usepackage{hyperref}
%\usepackage{ctex}

%SetFonts

%SetFonts

\usepackage[inline]{asymptote}


\pagestyle{fancy}
\fancyhf{}
\rhead{Zifei (David) Zhong, \url{zhongz@email.sc.edu}}
\lhead{\leftmark}
\lfoot{\href{https://github.com/zfzhong/csce791}{github.com/zfzhong/csce791}}

\title{CSCE 791 Course Report}
\author{Zifei (David) Zhong}
%\date{}							% Activate to display a given date or no date

\newcommand{\latex}{\LaTeX\xspace}


\begin{document}
\maketitle

This file includes all my course reports for CSCE 791 for the semester of Fall 2022. I will write a summary for each talk presented in the seminar. I will also include interesting questions raised by the audience, as well as corresponding responses from the speaker.\\

\begin{center}
\begin{tabularx}{0.65\textwidth}{r X}
\textbf{Course:} & CSCE 791: Seminar in Advances in Computing\\
\textbf{Location:} & Storey Innovation Center 1400\\
\textbf{Time:} & Friday 2:20pm - 3:10pm\\
\textbf{Semester:} & Fall 2022
\end{tabularx}
\end{center}

\newpage
\section{Talk on August 26, 2022}
I didn't attend the talk, due to my improper management of time and meetings.

\newpage
\section{Talk on September 2, 2022}
\begin{tabularx} {\textwidth}{r X}
\textbf{Topic}: & AI for science:  How machine learning and deep learning are transforming materials discovery \\
\textbf{Speaker:} & Dr. Jianjun Hu \\\\
\textbf{Abstract:} & Artificial intelligence and deep learning are increasingly transforming all scientific disciplines with their superior capability to learn to detect patterns from large amount of data and to learn predictive models from  data without relying upon theory or deep mechanistic understanding, with their capability to build generative models for inverse design of materials and molecules and with the models to generate synthetic data. In this talk, we present our research focusing on using deep learning and machine learning to discover and model the patterns in and relationships of structures and functions in materials and molecules and how to exploit such learned dark/implicit knowledge in deep learning based generative design of novel materials, graph neural network based materials property prediction, and deep learning based crystal structure prediction of inorganic materials. Considering that the number of inorganic materials discovered so far (~250,000) by humanity is only a tiny portion of the almost infinite chemical design space, our AI based data-driven computational materials discovery has the potential to transform the conventional trial-and-error approaches in materials discovery. \\
\end{tabularx}

\end{document} 