\documentclass[11pt, oneside]{article}   	% use "amsart" instead of "article" for AMSLaTeX format
%\usepackage{geometry}                		% See geometry.pdf to learn the layout options. There are lots.
%\geometry{letterpaper}                   		% ... or a4paper or a5paper or ... 
%\geometry{landscape}                		% Activate for rotated page geometry
%\usepackage[parfill]{parskip}    		% Activate to begin paragraphs with an empty line rather than an indent

\usepackage{geometry}
 \geometry{
 a4paper,
 total={170mm,257mm},
 left=20mm,
 top=22mm,
 bottom=22mm
 }

\usepackage{graphicx}				% Use pdf, png, jpg, or eps§ with pdflatex; use eps in DVI mode
								% TeX will automatically convert eps --> pdf in pdflatex		
\usepackage{amssymb}
\usepackage{amsmath}
\usepackage{fancyhdr}
\usepackage[utf8]{inputenc}
\usepackage[english]{babel}
\usepackage{enumerate}
\usepackage{arcs}
\usepackage{cancel}
\usepackage{xfrac}
\usepackage{amsthm}
\usepackage{gensymb}
\usepackage{xspace}
\usepackage{array}
\usepackage{tabularx}
\usepackage{url}
\usepackage{hyperref}
%\usepackage{ctex}

%SetFonts

%SetFonts

\usepackage[inline]{asymptote}


\pagestyle{fancy}
\fancyhf{}
\rhead{Zifei (David) Zhong, \url{zhongz@email.sc.edu}}
\lhead{\leftmark}
\lfoot{\href{https://github.com/zfzhong/csce791}{github.com/zfzhong/csce791}}
\rfoot{\thepage}

\title{CSCE 791 Course Report}
\author{Zifei (David) Zhong}
%\date{}							% Activate to display a given date or no date

\newcommand{\latex}{\LaTeX\xspace}


\begin{document}
\maketitle

This file includes all my course reports for CSCE 791 for the semester
of Fall 2022. I will write a summary for each talk presented in the
seminar. I will also include interesting questions raised by the
audience, as well as corresponding responses from the speaker.\\

\begin{center}
\begin{tabularx}{0.65\textwidth}{r X}
\textbf{Course:} & CSCE 791: Seminar in Advances in Computing\\
\textbf{Location:} & Storey Innovation Center 1400\\
\textbf{Time:} & Friday 2:20pm - 3:10pm\\
\textbf{Semester:} & Fall 2022
\end{tabularx}
\end{center}

\newpage
\section{Talk on August 26, 2022}
I didn't attend the talk, due to my improper management of time and meetings.

\newpage
\section{Talk on September 2, 2022}
\begin{tabularx} {\textwidth}{r X}
\textbf{Topic}: & AI for science:  How machine learning and deep
learning are transforming materials discovery \\
\textbf{Speaker:} & Dr. Jianjun Hu \\
\end{tabularx}

\subsection{Review}
Dr. Hu presented his work of using AI tools (GAN, Transformer, etc) to
learn the structures/features of materials and molecules.  He
presented his work on three subjects: 1) generative design of material
composition, 2) generative design of cubic crystal structures, and 3)
generative design of molecules.


He presented the \emph{one-hot matrix} as an effective numeral
representation of known materials. On top of that he applied GAN to
generate new matrices that make senses. These generated material
should chemically make sense (bounded by many chemical rules). He also
presented the way of validating his results.


He also presented many details of his work on design of cubic crystal
structures and design of molecules. He also mentioned an algorithm
he borrowed from text generation and applied to design of
molecules.


His results on these subjects are remarkable. I believe pushing his
discovery to help develop new materials might make great impact in the
future.



\subsection{Q\&A}
At the end of the talk, I asked two questions: 
\begin{enumerate}
\item I pointed out that the one-hot matrices he developed as numeral
  representation of material structure is most likely sparse, and
  asked him the possibility of using simple ways (for example, some
  kind of linear combination of a series of matrices) to generate new
  matrices that make senses. He responded that it's a great insight to
  realize the matrices are sparse, and other researchers tried some
  different ways but they did not get meaingful results.
  
\item I asked the possibility of extending his work as a useful tool
  for material scientists to develop new materials. He responded that
  it would be his next-step work.
  
\end{enumerate}

%\subsection{My Draft Notes Written in Class}
%climate change, energy crisis, cancers, diseases, 
%
%
%look for energy efficient materials
%
%look for better drugs
%
%alphafold 2
%
%
%introducing the achievements  of deep learning in AI, without
%manually figuring out features.

%
%how to use DL to design new materials
%
%four fundamental problems of science: materials, chemistry, physics,
%biology

%
%predication with large data vs. experimental process 
%
%
%the challenges of materials design: space groups, a lot of rules
%
%
%rational design vs generative design
%
%3 stories (3 research problems): generative design of material
%composition, ...

%
%
%1. design of material composition: 
%dataset: 120000 known materials formula from materials project database
%material formula generation $\rightarrow$ one-hot matrix generation
%
%120000 matrices to generate a new matrix that makes sense (use GAN)
%
%validity of generated materials (chemically reasonable)
%
%2. generative design of cubic crystal structures
%
%dataset: 388680 known cubic materials structures from OQMD database
%
%cubicGAN
%
%
%validated by DFT calculations
%
%carolinamatdb.org
%
%3. generative design of molecules  (transformer language model for
%sequence modeling and generation)

%
%similar to reinforcement learning?
%
%attention mechanism
%
%molecules generation as a canvas rewriting process
%
%borrow algorithm from MIT text generation paper
 
%Question: I point out that most of the one-hot matrices should be
%sparse matrices, and ask will linear combination of those matrices
%generate matrices that make sense? He responded that's a good insight
%and continued to mention that there were other similar ways but his
%way is the best in terms of getting results that make senses.


%I also suggest that this work can be a useful to guide material
%scientists to develop new materials. He agrees and says that's the
%next step.

\newpage
\section{Talk on September 9, 2022}
\begin{tabularx} {\textwidth}{r X}
\textbf{Topic}: & AI $\times$ Mathematics \\
\textbf{Speaker:} & Dr. Petar Veličković\\
\end{tabularx}

\subsection{Review}
Dr. Veličković presented his work that applies machine learning tools
to solve difficult problems in mathematics. His team has been working
on this project for about 1.5 years.


He gave a brief introduction on Coxeter Group theory and introduced
the Combinatorial Invariance Conjecture, which has stood for about 50
years. They employed the Graph Neural Network (GNN) and trained the
model over dataset of intervals up to $S_9$, and the GNN learned to
predict KL coefficients significantly better than chance. Using
gradient saliency to identify salient nodes and edges by the GNN, they
improved the model and got more than 95\% accuracy in predicting KL
coefficients.


With a well understanding of the model, they developed \emph{hypercube
decomposition} of the Bruhat interval, and found it very promising for
inductively computing the KL coefficients. Based on their experimental
results, they proposed a new conjecture which is waiting for proof.


\subsection{Q \& A}

%I am too old to educate anyone. But when I was a PhD student in UT
%Austin many years ago, I was advised to ask at least one question
%(even silly ones) for any talk I attend. Even though I ended up
%dropping out from the program,  I always keep that in mind when I
%walk into a talk conference room -- what is the question that I am
%going to ask?


At the end of the talk, I asked one question:
\begin{enumerate}
\item I noticed that for the case $q$, their prediction achieves
  99.9\% accuracy. I believe they examined the instances of the 0.1\%
  which were not covered by their prediction. I asked: were these
  instances (among the 0.1\%) of significant importance?
  

He responded ``yes". He even disclosed that examining these instances
helped them develop the hypercube decomposition (if I understand
correctly).


\end{enumerate}

\newpage
\section{Talk on September 16, 2022}
\begin{tabularx} {\textwidth}{r X}
\textbf{Topic}: & Can We Ever Trust Our Chatbots? Towards Trustable Collaborative Assistants\\
\textbf{Speaker:} & Dr. Biplav Srivastava\\
\end{tabularx}

\subsection{Review}
Dr. Srivastava introduced chatbot related research in his talk. He presented chatbots as collaborative assistants for decision support system that engage one or more people in conversation. He also presented the conceptual architecture of chatbots -- a data-driven dialog system. He showed some case studies of chatbots, and then pointed out the problem with trust for chatbots.

For the problem of trust, he discussed several components of the issue: competent, reliable, social values, and human-technology interaction, etc. These discussions gave the audience better understanding of the issue.

In the last part, he discussed some possible ways of making chatbots more trustable. He covered testing, AI robustness, randomized control trial and the idea of rating AI based on their behavior. I believe these means generally make sense, but I also should acknowledge that many of these are beyond my expertise.

\subsection{Q \& A}
I wanted to ask some questions but only to found out that I was not in the right position to ask good questions that make senses. 

% \subsection{Notes}
% history of chatbots is the history of ai, amazon alexa, google echo, apple siri, ...
% 
% collaborative assistants for decision support 
% system that engage one or more people in conversations
% 
% conceptual architecture: a data-driven dialog system
% 
% characteristics and potential 
% - retrieve information
% - help in decision making
% 
% 
% extensive research experience with chatbots
% 
% promise of chatbots
% 
% case study: ai and productivity 
% 
% medicare insurance recommendation 
% 
% kite - unsupervised content exploration 
% 
% problem with trust 
% 
% what are the components of trust 
% - competent
% - reliable
% - upholds human values, social good
% - allow human-technology interaction 
% 
% 
% current ai: capability, limitation, ethical issues
% 
% ai-based decision support for covid-19
% 
% chatbots during covid-19 - gaps
% - inconsistent ability
% - missing differentiations over alternatives 
% - inaccessible information to many users
% - ambiguity regarding user privacy
% 
% chatbots - recommendation to fill gaps 
% 
% transparency through documentation of rating 
% 
% labels help consumers make informed decisions about food
% 
% problem we are tackling for ai
% 
% collaborative assistant - chatbots



\newpage
\section{Talk on September 23, 2022}
\begin{tabularx} {\textwidth}{r X}
\textbf{Topic}: & The Gregarious Machines\\
\textbf{Speaker:} & Dr. Amitava Das\\
\end{tabularx}

\subsection{Review}
Dr. Amitava Das gave a talk on human behavior and personality study. He believes that such study is important for the development of artificial intelligence. 

He showed some interesting results from his research about using machine learning tools to analyze human characteristics. For example, it's interesting to learn how he defines \textit{community} of human society (According to Dr. Das, \emph{Communities} are dense sub-networks within a loose-connected network). Dr. Das also introduced his study on hate speech, and he showed that a group of people with certain characteristics are likely to generate hate speech. 

Generally, he gave the high level ideas without going to the technical details of applying machine learning tools. He also showed some videos to demonstrate the power of machine learning applied to studying human behaviors and characteristics.

\subsection{Q \& A}
At the end of the talk, I asked one question: 
\begin{enumerate}
\item When identifying the group of people who are likely to generate hate speech, is the environment that an individual stays in an important factor that is likely to affect him/her to behave? \\ \\
He said ``yes''. He mentioned that a certain result might come from the joint effect of human personality and the community environment where people stay in. His study on the \textit{communities} of human society is to address causes from the environment aspects. 
\end{enumerate}
 
\newpage
\section{Talk on October 7, 2022}

\begin{tabularx} {\textwidth}{r X}
\textbf{Topic}: & Logic meets Learning: From Aristotle to Neural Networks\\
\textbf{Speaker:} & Dr. Vaishak Belle\\
\end{tabularx}

\subsection{Review}
Dr. Belle holds the view that logic reasoning can be used to extend the boundaries of machine learning. In this talk, he gave several examples where logic reasoning and learning are combined together. I roughly got the high level ideas of these examples without understanding them in depth. 

Dr. Belle also listed several research areas where logic reasoning is widely used in connection to learning. Logic can help learning in many ways, while learning can address the fundamental equation of how to arrive at symbolic knowledge. 

I believe logic reasoning can be more widely used in learning, and learning can help disclose the logic connection in data as well.


\subsection{Q \& A}
At the end of the talk, I asked one question:
\begin{enumerate}
  
\item Since many researchers are trying to use machine learning tools to solve math test problems, I think it is a topic that logic reasoning should be combined with machine learning. I asked him for comments. I asked the question simply because a machine can not verify the correctness of its solution to a math problem without knowing the math logic behind that problem.\\ \\
In his response, I believe that he talked about how it's been done: giving the machine a set of examples to train a solver, and then applying the solver to another set of problems. He also talked about the pros and cons.\\


\end{enumerate}


\newpage
\section{Talk on October 21, 2022}

\begin{tabularx} {\textwidth}{r X}
\textbf{Topic}: & Harnessing Mean Field Game and Physics-Informed Deep Learning for Emerging Transportation Modeling\\
\textbf{Speaker:} & Dr. Xuan (Sharon) Di \\
\end{tabularx}

\subsection{Review}
Dr. Di introduced her work on designing optimal controls for autonomous vehicles. There are several levels of abstraction in her work. First, she used certain models to represent real world problems with traffic and routing assumptions. Second, she formulated optimization problems based on these models, and solved these optimization problems subsequently. 

Later she introduced her work that employed machine learning tools to solve transportation problems. On particular point is that she believes incorporating physical logic into machine learning will help achieve better results. She evaluated her work with real traffic data. 

He talk was very theoretical with many formulas and equations. The high level ideas of her work were easy to understand, but I could not really get these math abstraction and modeling. 



\subsection{Q \& A}
Since her model involves traffic balancing and best route selection, I asked if she considered extreme scenarios where a single factor (distance or time, for example) might be significantly important than any other factors. 

She said that was already taken care by her model, because in her model the weight on each factor is adjustable. If she wants to favor one factor, she will just increase the weight of that factor in her model.

\end{document} 
