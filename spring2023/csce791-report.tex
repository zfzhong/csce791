\documentclass[11pt, oneside]{article}   	% use "amsart" instead of "article" for AMSLaTeX format
%\usepackage{geometry}                		% See geometry.pdf to learn the layout options. There are lots.
%\geometry{letterpaper}                   		% ... or a4paper or a5paper or ... 
%\geometry{landscape}                		% Activate for rotated page geometry
%\usepackage[parfill]{parskip}    		% Activate to begin paragraphs with an empty line rather than an indent

\usepackage{geometry}
 \geometry{
 a4paper,
 total={170mm,257mm},
 left=20mm,
 top=22mm,
 bottom=22mm
 }

\usepackage{graphicx}				% Use pdf, png, jpg, or eps§ with pdflatex; use eps in DVI mode
								% TeX will automatically convert eps --> pdf in pdflatex		
\usepackage{amssymb}
\usepackage{amsmath}
\usepackage{fancyhdr}
\usepackage[utf8]{inputenc}
\usepackage[english]{babel}
\usepackage{enumerate}
\usepackage{arcs}
\usepackage{cancel}
\usepackage{xfrac}
\usepackage{amsthm}
\usepackage{gensymb}
\usepackage{xspace}
\usepackage{array}
\usepackage{tabularx}
\usepackage{url}
\usepackage{hyperref}
%\usepackage{ctex}

%SetFonts

%SetFonts

\usepackage[inline]{asymptote}


\pagestyle{fancy}
\fancyhf{}
\rhead{Zifei (David) Zhong, \url{zhongz@email.sc.edu}}
\lhead{\leftmark}
\lfoot{\href{https://github.com/zfzhong/csce791}{github.com/zfzhong/csce791/spring2023}}
\rfoot{\thepage}

\title{CSCE 791 Course Report}
\author{Zifei (David) Zhong}
%\date{}							% Activate to display a given date or no date

\newcommand{\latex}{\LaTeX\xspace}


\begin{document}
\maketitle

This file includes all my course reports for CSCE 791 for the semester
of Spring 2023. I will write a summary for each talk presented in the
seminar. I will also include interesting questions raised by the
audience, as well as corresponding responses from the speaker.\\

\begin{center}
\begin{tabularx}{0.65\textwidth}{r X}
\textbf{Course:} & CSCE 791: Seminar in Advances in Computing\\
\textbf{Location:} & Swearingen Engineering Center 2A17\\
\textbf{Time:} & Friday 2:20pm - 3:10pm\\
\textbf{Semester:} & Spring 2023
\end{tabularx}
\end{center}

\newpage
\section{Talk on January 18, 2023}
\begin{tabularx} {\textwidth}{r X}
\textbf{Topic}: & From Self-Adaptation to Self-Evolution \\
\textbf{Speaker:} & Danny Weyns \\
\end{tabularx}

\subsection{Review}
In this talk, Dr. Weyns introduced the self-adaptation concept of computer system and the conceptual model of self-adaptive system in general. These were very abstract concepts that require audience to have proper understanding of software engineering or architecture. He then brought up the question why ``just self-adaptation'' is not enough. He introduced the Operational Design Domain (ODD) of a system, and claimed that evolution of a system means enhancing the ODD of a system. He gave an example IoT system to explain how to enhance the ODD of a system. To enhance the ODD, updating current software is required, and all that requirements point to a new software architecture. Finally, he introduce the architecture of a system with evolution as the ultimate solution. The new architecture has two new components: computing warehouse and data storage, which are two major components which support evolutional computing. He also talked about the challenges to realize such system architecture.

The talk was well structured and it flowed naturally, even though some concepts were quite abstract at the beginning. The ODD concept is a great abstraction, and the new architecture to support software evolution seems reasonable. The new architecture proposal can be viewed as a design to install a brain for a computer system. I expect to know, in the near future, what rules engineers should follow to install such brains for future computer systems.

\subsection{Questions}
I didn't ask a question because I had conflict class schedule (CSCE 574), for which I was not able to attend the talk in-person.


\end{document} 
